\section{Įvadas}

\par \emph{Išdėstomi praktikos vietos pasirinkimo motyvai, praktikos
  užduotis, jos tikslas, spręstieji uždaviniai, pateikiama praktinės
  veiklos planas, praktikos atlikimo eiga (2-3 psl.)}


\section{Įmonės apibūdinimas}

\par \emph{Glaustai aprašoma įmonė, kurioje buvo atliekta praktika:
  jos veiklos sritis, organizacinė struktūra, teikiamos paslaugos ir
  kt. apibūdinamos praktikos vietoje sudarytos darbo sąlygos
  (1-2psl.)}


\section{Praktikos veiklos aprašymas}

\par \emph{(vienas arba keli skyriai). Aprašomas praktikos užduoties
  įgyvendinimas (pvz., atlikti projektavimo ir/ar programavimo darbai,
  sukurtas modelis, priimti sprendimai ir pan.).}

\subsection{Skyrius 1}

\subsubsection{Gilus poskyris 1.1}

\subsubsection{Gilus poskyris 2.1}

\subsection{Skyrius 2}


\section{Rezultatai, išvados ir pasiūlymai}

\par \emph{Išdėstomi pagrindiniai darbo rezultatai ir išvados,
  praktikos darbo privalumai ir trūkumai, aprašomos įgytos žinios ir
  patirtis praktikos metu, duodamas universitete įgytų žinių atitikimo
  praktikos užduočiai atlikti įvertinimas, pateikiami argumentuoti
  pasiūlymai, kaip geriau organizuoti darbo ir valdymo procesus
  praktikos atlikimo vietoje ir mokymą Universitete (1-2 psl.).}
